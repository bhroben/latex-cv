%----------INTESTAZIONE----------%
\begin{center}
    \textbf{\Huge \scshape Roben Bhatti} \\ \vspace{1.5pt}
    \href{mailto:robenbhatti@gmail.com}{\seticon{faEnvelope} robenbhatti@gmail.com} \quad
    \href{https://www.linkedin.com/in/roben-bhatti}{\seticon{faLinkedin} linkedin.com/roben-bhatti} \quad
    \href{https://github.com/bhroben}{\seticon{faGithub} github.com/bhroben} \quad
    \\ \vspace{2pt}
    \seticon{faPhone} \ \small +39 331 211 6804 \quad 
    \seticon{faMapMarker} \ \small Brescia, Italia \quad 
    \seticon{faFlag} \small Cittadinanza italiana \quad 
\end{center}

%-----------ESPERIENZA-----------%
\section{Esperienza}
\resumeSubHeadingListStart

    \resumeSubheading
    {Agenzia Spaziale Tedesca (DLR)}{Ott 2024 - Luglio 2025}
    {Data Scientist Intern}{Brema, Germania}
    \resumeItemListStart
        \resumeItem{Sviluppato e implementato modelli bayesiani per quantificare l'incertezza nei coefficienti aerodinamici di veicoli spaziali riutilizzabili.}
        \resumeItem{Sviluppato una pipeline \textbf{CI/CD} con \textbf{Gitlab} e \textbf{Docker}, automatizzando il processo di release garantendo affidabilità.}
        \resumeItem{Collaborato in un team \textbf{Agile} ottenendo risultati basati sui dati, portando a termine la mia Tesi Magistrale e a un articolo \textbf{co-autore} per una conferenza (IAC 2025) (in preparazione).}
        \resumeItem{Ridotto dei bottleneck computazionali nei modelli grazie all'implementazione di Gaussian Processes Sparsi, con una \textbf{riduzione del 30\%} dei tempi di addestramento e un’analisi dell’incertezza più scalabile.}
    \resumeItemListEnd

\resumeSubHeadingListEnd

%-----------ISTRUZIONE-----------%
\section{Formazione}
    \resumeSubHeadingListStart

    \resumeSubheading
    {Università di Padova}{Ott 2022 - Lug 2025}
    {Laurea Magistrale in Fisica "Physics of Data" (Votazione 110/110)}{Padova, Italia}
    \resumeItemListStart
        \resumeItem{\textbf{Corsi rilevanti:} Metodi Matematici e Numerici, Deep Learning e Reti Neurali, Statistica Avanzata per l'Analisi Fisica, Teoria dell'Informazione e Inferenza, Relatività Generale.}
    \resumeItemListEnd
    
    
    \resumeSubheading
    {Università di Padova}{Ott 2019 - Ott 2022}
    {Laurea Triennale in Astronomia (Votazione 95/110)}{Padova, Italia}
    \resumeItemListStart
        \resumeItem{\textbf{Corsi rilevanti:} Analisi Matematica Avanzata, Statistica, Meccanica Analitica, Fisica Quantistica, Relatività Ristretta.}
    \resumeItemListEnd

    \resumeSubHeadingListEnd

%-----------PROGETTI-----------%
\section{Progetti}
\resumeSubHeadingListStart

    \resumeProjectHeading
    {\textbf{Elaborazione di Dati di Fisica delle Particelle in Streaming con Spark e Kafka}}{}
    \resumeItemListStart
        \resumeItem {Progettazione di una pipeline distribuita in grado di elaborare oltre \textbf{10 GB} di dati al giorno, simulando lo streaming in tempo reale di un rivelatore di particelle tramite \textbf{Kafka} e \textbf{Spark}.}
        \resumeItem{Deploy dell'ambiente su cluster cloud con \textbf{AWS S3} per lo storage, Kafka per la messaggistica, e Spark per l’elaborazione distribuita.}
        \resumeItem{Creazione di una dashboard interattiva in tempo reale con Bokeh per la \textbf{visualizzazione dei dati}.}
    \resumeItemListEnd

    \resumeProjectHeading
    {\textbf{Riconoscimento di Posizioni di Scacchi con Transformers (DETR) in PyTorch}}{}
    \resumeItemListStart
        \resumeItem{Sviluppo di una pipeline di computer vision per digitalizzare partite fisiche di scacchi, con una precisione dell'\textbf{87.5\%} nel riconoscere pezzi e posizioni da un'unica immagine.}
        \resumeItem{Ottimizzazione del modello DETR (DEtection TRansformer) tramite \textbf{transfer learning}, riducendo significativamente i tempi di addestramento e migliorando le prestazioni su dataset personalizzati.}
        \resumeItem{Sviluppo di un modulo di \textbf{post-processing} per convertire automaticamente le rilevazioni grezze in notazione standard (FEN), rendendo subito compatibile lo stato della scacchiera con motori scacchistici e strumenti di analisi.}
    \resumeItemListEnd

    \resumeProjectHeading
    {\textbf{Pipeline di Machine Learning End-to-End per il Profiling di Clienti Assicurativi} (In corso)}{}
    \resumeItemListStart
        \resumeItem{Implementazione di un'architettura "Medallion" per garantire qualità e scalabilità su un dataset di \textbf{20K} polizze assicurative.}
        \resumeItem{Sviluppo di un modello in due fasi per prevedere gli importi dei sinistri: classificazione della probabilità di sinistro, seguita da una regressione con Random Forest per stimare l’ammontare nei casi ad alta probabilità.}
    \resumeItemListEnd

\resumeSubHeadingListEnd

%-----------COMPETENZE TECNICHE-----------%

\section{Competenze Tecniche}
    \begin{itemize}[leftmargin=0.15in, label={}]
	\small{\item{
		\textbf{Linguaggi}{: Python, R, SQL} \\
		\textbf{Tecnologie}{: Docker, Git, CI/CD, Kafka, Spark, Pytorch, SciPy, NumPy, Pandas, Scikit-learn, Tableau.} \\}}
    \end{itemize}

\section{Lingue}
     \begin{itemize}[leftmargin=0.15in, label={}]
 	\small{\item{
 		\textbf{Inglese}: C1, \textbf{Italiano}: Madrelingua }}
     \end{itemize}
