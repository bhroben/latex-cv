%%%%%%%%%%%%%%%%%%%%%%%%%%%%%%%%%%%%%%%%%
% Developer CV
% LaTeX Class
% Version 2.0 (12/10/23)
%
% This class originates from:
% http://www.LaTeXTemplates.com
%
% Authors:
% Omar Roldan
% Based on a template by  Jan Vorisek (jan@vorisek.me)
% Based on a template by Jan Küster (info@jankuester.com)
% Modified for LaTeX Templates by Vel (vel@LaTeXTemplates.com)
%
% License:
% The MIT License (see included LICENSE file)
%
%%%%%%%%%%%%%%%%%%%%%%%%%%%%%%%%%%%%%%%%%

%----------------------------------------------------------------------------------------
%	PACKAGES AND OTHER DOCUMENT CONFIGURATIONS
%----------------------------------------------------------------------------------------

\documentclass[9pt]{developercv} % Default font size, values from 8-12pt are recommended
\usepackage{multicol}
\setlength{\columnsep}{0mm}
%----------------------------------------------------------------------------------------
\usepackage{lipsum}  


\begin{document}

%----------------------------------------------------------------------------------------
%	TITLE AND CONTACT INFORMATION
%----------------------------------------------------------------------------------------

\begin{minipage}[t]{0.5\textwidth} 
	\vspace{-\baselineskip} % Required for vertically aligning minipages
	
	{ \fontsize{16}{20} \textcolor{black}{\textbf{\MakeUppercase{Roben Bhatti}}}} % First name
	
	\vspace{6pt}
	
	{\Large M.Sc Physics of Data Student} % Career or current job title


\end{minipage}
\hfill
\begin{minipage}[t]{0.2\textwidth} % 20% of the page width for the first row of icons
	\vspace{-\baselineskip} % Required for vertically aligning minipages
	
	% The first parameter is the FontAwesome icon name, the second is the box size and the third is the text
	%\icon{Globe}{11}{\href{http://www.google.com}{portafolio.com}}\\ 
    \icon{Linkedin}{11}{\href{https://www.linkedin.com/in/roben-bhatti/}{in/roben-bhatti}}\\
    \icon{MapMarker}{11}{Padua, Italy}\\
	\icon{Phone}{11}{+39 331 211 6804}\\
\end{minipage}
\begin{minipage}[t]{0.27\textwidth} % 27% of the page width for the second row of icons
	\vspace{-\baselineskip} % Required for vertically aligning minipages
	
	\icon{Envelope}{11}{\href{mailto:robenbhatti@gmail.com}{robenbhatti@gmail.com}}\\	
    \icon{Github}{11}{\href{https://github.com/bhroben}{github.com/bhroben}}\\
    %\icon{LinkedinSquare}{11}{\href{https://www.linkedin.com}{/in/your-personal-url}}\\    
    \icon{Flag}{11}{Italian Citizenship}\\
\end{minipage}

\hfill

%----------------------------------------------------------------------------------------
%	INTRODUCTION, SKILLS AND TECHNOLOGIES
%----------------------------------------------------------------------------------------

\begin{minipage}[t]{0.46\textwidth}
    \cvsect{Summary}
	\vspace{-6pt}
 
    %Dummy text
    M.Sc. Physics of Data student with a solid foundation in physics, mathematics, and statistics. Passionate about applying analytical and computational skills to solve real-world problems.
 
\end{minipage}
\hfill % Whitespace between
\begin{minipage}[t]{0.465\textwidth}
    \cvsect{Skills}
    \vspace{-6pt}
    
    \begin{minipage}[t]{0.2\textwidth}
        \textbf{Languages:}
    \end{minipage}
    \hfill
    \begin{minipage}[t]{0.73\textwidth}
      Python, R, SQL, Shell.  
    \end{minipage}
    \vspace{4mm}
    
    \begin{minipage}[t]{0.2\textwidth}
        \textbf{Technologies:}
    \end{minipage}
    \hfill
    \begin{minipage}[t]{0.73\textwidth}
      Docker, Git, CI/CD, Anaconda, Kafka, Spark, Keras, Pytorch, SciPy, NumPy, Pandas, Scikit-learn, XGBoost.
    \end{minipage}
    
\end{minipage}

%----------------------------------------------------------------------------------------

%-----------------------------------------------------------------
%	EXPERIENCE
%----------------------------------------------------------------------------------------
\vspace{-10 pt}
\cvsect{Experience}
\begin{entrylist}
	\entry
        {10/2024 -- now}
		{Data Scientist Intern}
		{German Aerospace Center (DLR) in Bremen (DE)}
            {- Developed a Bayesian Framework for uncertainty estimation of Aerodynamic Coefficients. 
            
            - Set up CI/CD pipeline, Unit Tests, Linting, and modular package structure following PEP8.
            
            - Applied sparse methods for efficient Bayesian computation.
            
            - Followed Scrum Workflow for structured and iterative development.}
		
\end{entrylist}

%----------------------------------------------------------------------------------------
%	EDUCATION
%----------------------------------------------------------------------------------------
\vspace{-10 pt}
\cvsect{Education}
\begin{entrylist}
    \entry
		{10/2022 - 07/2025}
		{Master Degree in Physics of Data }
		{University of Padua}
		{Master degree program that merges and innovates the educational offers from Physics and Data Science.}
    \entry
		{10/2019 - 10/2022}
		{Bachelor Degree in Astronomy}
		{University of Padua}
		{Bachelor program provides solid foundation in physics, mathematics, and statistics.}

\end{entrylist}

%----------------------------------------------------------------------------------------

%	Projects
%----------------------------------------------------------------------------------------
\cvsect{Projects}
\begin{entrylist}
    
    \entry
        {}
        {\href{https://github.com/XRD-Group/End-To-End-ML-pipeline-for-profiling-insurance-customers}{End-to-End ML Pipeline for Profiling Insurance Customers}}
        {PostgreSQL, Python, CI/CD, Docker}
        {Built a medallion-architecture Data Lakehouse in PostgreSQL and predicted insurance claim amounts per customer using Random Forest model.}
    \entry
		{}
		{\href{https://github.com/bhroben/Streaming-processing-of-cosmic-rays-using-drift-tubes-detectors}{Streaming processing of cosmic rays using drift tubes detectors}}
		{Kafka, PySpark, SQL, Docker}
		{Designed and implemented a real-time ETL pipeline to process data from a particle physics detector, publishing analytics to a live dashboard for continuous monitoring.}
    \entry
		{}
		{\href{https://github.com/bhroben/Bayesian-Optimization-with-Gaussian-Process}{Bayesian optimization with Gaussian Processes}}
		{Python, TensorFlow}
		{Implemented Gaussian Process models for hyperparameter optimization of convolutional neural networks (CNNs). Explored both MCMC and maximum likelihood approaches for tuning advanced kernel parameters.}
    \entry
        {}
        {\href{https://github.com/bhroben/DETR-for-recognition-of-real-chess-game}{DETR for recognition of real chess game }}
        {Pytorch}
        {
            Developed and fine-tuned a Transformer-based model to accurately identify chess pieces and their positions on real chessboards. Automated the conversion of board states into standard FEN notation for further analysis.}
    \entry
		{}
		{\href{https://github.com/bhroben/Feature-importance-methods-of-simulated-binary-black-holes}{Feature importance methods of simulated binary black holes}}
		{Python, Machine Learning}
        {Analyzed key features influencing the evolution of binary systems into Binary Black Holes using advanced machine learning techniques. Presented findings to highlight the most impactful variables in the process.}
	\entry
		{}
		{\href{https://github.com/bhroben/Naive-Bayes-multinomial-classifier-for-fake-news-detection}{Naive Bayes multinomial classifier for fake news detection}}
		{R}
		{%Dummy text 
        Accurate and automated identification of fake news sentences using Bayes Theorem.}


\end{entrylist}

%-----------------------

%	LANGUAGES
%----------------------------------------------------------------------------------------
\vspace{-10 pt}
	\cvsect{Languages}
    \vspace{-6pt}
    
    \hspace{26mm} \textbf{English} - C1, 
    \textbf{ Italian} - native

%----------------------------------------------------------------------------------------
%----------------------------------------------------------------------------------------
%	Extra
%----------------------------------------------------------------------------------------
\vspace{-10 pt}
\cvsect{Extra}
\begin{entrylist}

	\entry
	{11/2023}
	{NOI Hackaton SFSCON Edition}
	{Bolzano}
	{Developed an AI prototype during a 24-hour hackathon, leveraging computer vision to detect parking abuse, assist customers, and generate big data insights. Collaborated under tight deadlines, set clear goals, and delivered solutions effectively.}
 
 	\entry
        {3/2022 -- 6/2023}
		{Study Room surveillance}
		{University of Padua}
            {Provided assistance, resolved issues, and ensured a conducive environment.}

\end{entrylist}

\end{document}
